\documentclass[bigger, aspectratio=169]{beamer}
% In order to compile this document, you'll have to have the
% metropolis theme on your computer, as well as the adobe source code
% pro font. 
\beamerdefaultoverlayspecification{<+->}
\usetheme[titleformat=smallcaps, progressbar=frametitle]{metropolis}
\setmonofont{Source Code Pro}
\usepackage{xcolor}
\usepackage{graphicx}
\graphicspath{{images/}}
\usepackage[citestyle=authoryear]{biblatex}
\addbibresource{../bibliography/references.bib}

\usepackage{booktabs}

\title{Best Practices. For Science}
\author{J. Alexander Branham}

\usepackage{hyperref}

\begin{document}
{
  \usebackgroundtemplate{\includegraphics[width=\paperwidth]{for-science.jpg}}
  \begin{frame}
    \vspace{-2.5in} \hspace{2.75in} \Huge{\color{white} Best Practices.} 
  \end{frame}
}

\begin{frame}
  \frametitle{The Goal}
  \pause{}
  \centering
  \Huge{Get published}\\
  \pause{} \vspace{1in}
  \small{And do (good) science}
\end{frame}

\begin{frame}
  \frametitle{Why care?}
  Why care about all this reproducibility/best practices stuff? 
  \pause{}
  \begin{itemize}
  \item Journals require it
  \item It makes your life easier
  \item And for science 
  \end{itemize}
\end{frame}

\begin{frame}
  \frametitle{What will it do for me?}
  \pause{}
  \begin{itemize}
  \item Reduce errors
  \item Speed up re-running analyses
  \item Allow ourselves/others to reproduce our work in the future
  \item \url{https://youtu.be/s3JldKoA0zw?t=4s}
  \end{itemize}
\end{frame}

\begin{frame}
  \frametitle{Papers}
  A paper has at least some of:
  \begin{itemize}
  \item writing
  \item data \& summaries (tables, figures, statistics)
  \item citations (ugggghhh) \& cross-references
  \item appendix
  \end{itemize}
\end{frame}

\begin{frame}
  \frametitle{An ideal system}
  \begin{itemize}
  \item I want a system that \emph{keeps a record of my actions} as I:\@
    \begin{itemize}
    \item write \& cite
    \item analyze data
    \item present results
    \end{itemize}
  \item Reproducibly 
  \end{itemize}
\end{frame}

\begin{frame}
  In picking tools, some desirable qualities:
  \begin{itemize}
  \item Free (because we're poor grad students)
  \item Open source (so they'll be around later) 
    \begin{itemize}
    \item Quick moment of silence for Google Reader
    \end{itemize}
  \item Operate on \emph{plain text} (so they're usable in the future)
    \begin{itemize}
    \item Ever try to open a doc file from 10 years ago?
    \end{itemize}
  \item Work on most operating systems
  \item Don't hurt to learn
  \end{itemize}
\end{frame}

\begin{frame}
  \frametitle{Write \& cite}
  \begin{itemize}
  \item When writing Latex, you write in a plain-text file with a specific markup 
  \item It makes citations/bibliographies easy
  \end{itemize}
\end{frame}

\begin{frame}[fragile]
  \frametitle{Latex example}
  \begin{columns}
    \begin{column}{0.5\textwidth}
\begin{verbatim}
\section{Introduction}

This is a sentence! With 
\emph{emphasis}. 
And a 
citation \autocite{wlezien1995}. 

This is a new paragraph because 
there's a blank line. 
\end{verbatim}
    \end{column}
    \begin{column}{0.5\textwidth}
      This is a sentence! With \emph{emphasis}. 
      And a citation \autocite{wlezien1995}. 

      This is a new paragraph because there's a blank line. 
    \end{column}
  \end{columns}
\end{frame}

\begin{frame}[fragile]
  \frametitle{Bib example}
\begin{verbatim}
@article{wlezien1995,
  author = {Wlezien, Christopher},
  title = {The Public As Thermostat: Dynamics of 
             Preferences for Spending},
  journal = {American Journal of Political Science},
  volume = 39,
  pages = {981--1000},
  year = 1995,
  doi = {10.2307/2111666},
  url = {http://dx.doi.org/10.2307/2111666},
}
\end{verbatim}
\end{frame}

\begin{frame}[fragile]
  \frametitle{Latex table example}
\begin{verbatim}
  \begin{table}
    \centering
    \begin{tabular}{l r r}
           & model 1 & super improved model 2 \\
      \toprule
      IV1  & 0.2     & 0.1                    \\
           & (0.01)  & (0.05)                 \\
      IV 2 &         & 1.3                    \\
           &         & (0.7)                  \\
      \bottomrule
    \end{tabular}
    \caption{A pretty caption}\label{tab:best-table-ever}
  \end{table}
\end{verbatim}
\end{frame}

\begin{frame}
  \frametitle{Latex table example}
  \begin{table}
    \centering
    \begin{tabular}{l r r}
           & model 1 & super improved model 2 \\
      \toprule
      IV1  & 0.2     & 0.1                    \\
           & (0.01)  & (0.05)                 \\
      IV 2 &         & 1.3                    \\
           &         & (0.7)                  \\
      \bottomrule
    \end{tabular}
    \caption{A pretty caption}\label{tab:best-table-ever}
  \end{table}
\end{frame}


\begin{frame}
  \frametitle{Analyze data}
  \begin{itemize}
  \item R is free, open-source, and has the largest repository of statistical packages
  \item Python is also free and open source but has fewer users (in academic circles)
  \item Stata is probably the main competitor to R (in our discipline), but isn't free or open-source
  \end{itemize}
\end{frame}

\begin{frame}[fragile]
  \frametitle{Analyze data}
  \begin{itemize}
  \item Comment your code liberally
  \item If you write a function, test that it does what you think it does:
  \end{itemize}
\end{frame}

\begin{frame}[fragile]
  \frametitle{R example}
\begin{verbatim}
x <- rnorm(100) 
y <- rnorm(100, mean = 3 * x) # 100 draws from Y ~ N(3x, 1)
lm(y ~ x) # expect beta = 3ish
\end{verbatim}
\end{frame}

\begin{frame}[fragile]
  \frametitle{Test functions}
  \begin{verbatim}
timestwo <- function(x) {
  # take vector x and multiply each element by 2
  x + 2 
}
timestwo(-2:2) == c(-4, -2, 0, 2, 4)
\end{verbatim}

\end{frame}

\begin{frame}
  \frametitle{Reproduce}
  \begin{itemize}
  \item You write your paper, make your tables/figures, and ship it off to a journal.
  \item Eight months later, a reviewer wants you to tweak Figure 1.
  \item How did you make it in the first place? 
  \end{itemize}
\end{frame}

\begin{frame}
  \frametitle{Reproduce}
  \begin{itemize}
  \item One solution: \texttt{knitr}
  \item Allows you to use Latex and R in the same document
  \end{itemize}
\end{frame}

\begin{frame}[fragile]
  \frametitle{Rnw example}
\begin{verbatim}
\section{Introduction}

This is a sentence! With \emph{emphasis}. 
And a citation \autocite{wlezien1995}. 

<<R>>=
x <- rnorm(100) 
y <- rnorm(100, mean = 3 * x) # 100 draws from Y ~ N(3x, 1)
lm(y ~ x) # expect beta = 3ish
@
\end{verbatim}
\end{frame}

\begin{frame}[standout]
  Keep a record \\
  of \emph{everything} you do\\
\end{frame}

\begin{frame}[standout]
  \Huge{Keep a record} \\
  \Huge{of \emph{everything} you do}\\
\end{frame}

\begin{frame}
    \centering
    \includegraphics[width=0.5\textwidth]{final.jpg}
\end{frame}

\begin{frame}
  \frametitle{Git}
  \begin{itemize}
  \item Git keeps track of the whole project (data, writing, code) all in one place
  \item It allows you to comment on what's being changed in each \emph{commit}
  \item You can change things with no fear of losing anything
  \item Like ``track changes'' and backups on steroids
  \end{itemize}
\end{frame}

\begin{frame}
  \frametitle{Other tools}
  \begin{itemize}
  \item \texttt{make} --- for describing dependencies between files (especially useful for analyses that take longer to run)
  \item \texttt{pandoc} --- for when you need a docx version of your paper
  \end{itemize}
\end{frame}

{
  \usebackgroundtemplate{\includegraphics[width=\paperwidth]{aperture}}
  \begin{frame}
    And now for \\
    a live example\ldots
  \end{frame}
}

\end{document}
